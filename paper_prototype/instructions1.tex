\documentclass{article}
\begin{document}

\section*{Hanzi Hospital by michelleful}
\subsection*{Paper prototype \#1: Instructions}

\bigskip\bigskip

\textbf{What you'll need:}
\begin{itemize}
  \item A print-out of the prototype pages (3--8 inclusive)
  \item A good pair of scissors (lots of cutting!)
  \item A friend to act as the `computer' (preferable)
  \item Post-its (optional)
  \item A timer (optional)
\end{itemize}

\bigskip

\noindent\textbf{The backstory:}

 Times are tough in Hanzi City, where all the citizens are Chinese characters. Criminal elements have taken over the city and dangerous characters lurk in the darkness waiting to pounce on and attack unsuspecting civilians. Injured characters are brought to Hanzi Hospital for treatment. You are a trainee surgeon whose medical education has been interrupted to help deal with the influx of patients. Your job is to transplant components from volunteer characters into the wounded characters so that they can continue to live.
 
You're completely new to this, but luckily you're under the supervision of an experienced surgeon who practises her own brand of tough love. Initially, she arranges for all the donors available to you to be suitable for transplant and healing the patient. Later, you'll have to sort between the volunteer donors to figure out which one's the right one!

\bigskip

\noindent\textbf{How to operate the game:}

For each page (an individual `level'), the set-up process is as follows:

\begin{itemize}
  \item Cut out the `Donors' section of the page
  \item Cut out each individual donor (marked by the crudely-drawn pair of scissors)
  \item Cut the characters into components along the dotted lines and reassemble them as-is; alternatively, you can copy out the character onto a post-it note and cut it into two along the dotted-line, trying to make the character approximately the same size as the patient character.
  \item The `computer' should read the winning condition and then fold it over so it's not visible to the player
\end{itemize}

\newpage

When the player plays:
\begin{itemize}
  \item (Optional) start a timer for 30 seconds*
  \item When the player `clicks' on any of the donors, move the entire donor over to the `donor' bed under the patient.
  \item The player can now `drag and drop' any of the components over to fill in the blank space surrounded by dotted lines in the injured patient. Dropping the correct component in heals the patient -- the player beats the level and moves on to the next.
  \item If the player drags an incorrect component into the patient, it will be rejected. Sirens blare and their time is reduced by 5 seconds* (if playing on timer). Return the incorrect component to the donor and whisk the donor away -- they've suffered enough!
  \item If 30 seconds goes by without the player correctly `healing' the patient, an experienced surgeon steps in to heal the patient -- demonstrate it by moving the appropriate component over. (This shouldn't happen though, the first levels are really easy -- let me know if they're not!)
\end{itemize}

* To be adjusted through play-testing

\newpage

\setcounter{page}{9}

\subsection*{I've finished the playthrough...what now?}

\begin{itemize}
  \item You could pretend this is a classroom and you're reviewing what you've learned...what did the characters with shared components have in common? Discuss!
  \item Give feedback! How long did each level take you to solve? Was it too easy? Too boring? Would you prefer to start with something more challenging?  
\end{itemize}

\subsection*{What's next in the gameplan?}

\begin{itemize}
  \item Probably more at this level (unless feedback is that it's too boring), introducing:
    \begin{itemize}
      \item multi-character words
      \item more placement types -- my examples are mostly left-right for convenient cutting, but there are far more possible shapes.
      \item greater variation, especially on the phonetic level. All the examples I've shown so far share the same syllable with varying tone, but sometimes there are small variations in pronunciation. (In addition, there is *definitely* variation in meaning...not sure whether to address this for the more unreliable components or not.)
    \end{itemize}
  \item Once the player is comfortable with identifying phonetic/semantic similarities, the player is `promoted' one level and the experienced surgeon no longer arranges for all the donors to be suitable...only one is, and the player has to identify which one.
  \item As the player progresses, additional challenges come into play. More donors/distractors, more distractors with similar-looking components that you might mistake for the appropriate one...
  \item I'm thinking that in the final game there will be an interweave of narrative and gameplay. Most obviously, the player gets promoted up the food-chain in the hospital. But other things may occur as well and the tasks may change slightly from doing component transplants to, for example, cooperating with the police by identifying characters in `disguise'.
\end{itemize}


\end{document}